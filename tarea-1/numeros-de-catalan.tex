\documentclass[../portafolio.tex]{subfiles}
\begin{document}
\chapter{Numeros de catalán}
\chapterauthor{Antonio Sotomayor, Valentín Alarcón}
\hfill \textbf{Fecha de la actividad:} 8 de octubre de 2025

Sabiendo que 0! = 1
\begin{equation}
    C_0 = \frac{(2*0)!}{(0+1)!*0!} =\frac{1}{1} = 1     
\end{equation}
Para demostrar $C_{n+1}$ usamos lo siguiente:
\begin{equation}
    \frac{C_{n+1}}{C_n} = \frac{\frac{(2(n+1))!}{(n+2)!(n+1)!}}{\frac{(2n)!}{(n+1)!n!}}
\end{equation}
Como sabemos (n+1)! = n!(n+1) dando como resultado:

\begin{equation}
    \frac{C_{n+1} }{C_n}= \frac{a}{b}
\end{equation}


\end{document}

